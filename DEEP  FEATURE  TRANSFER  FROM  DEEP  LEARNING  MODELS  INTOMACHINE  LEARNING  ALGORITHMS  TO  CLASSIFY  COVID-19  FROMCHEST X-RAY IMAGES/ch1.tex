%%%%%%%%%%%%%%%%%%%%%%%%%%%%%%%%%%%%%%%%%%%%%%%%%%%%%%%%%%%%%%%%%
\chapter{INTRODUCTION}\label{CH1}
%%%%%%%%%%%%%%%%%%%%%%%%%%%%%%%%%%%%%%%%%%%%%%%%%%%%%%%%%%%%%%%%%
Coronaviruses are first named in early 1960s. Disease reports caused by this family of viruses had been reported about chickens since lately 1920s; however, human coronaviruses could be detected after 40 years. Then, because of their the distinctive morphological appearance, they were accepted as a new group of viruses and named as coronaviruses in 1968. First two human coronaviruses studied were named as coronavirus 229E and coronavirus OC43 \cite{history_coronavirus}. In following years, more coronaviruses have been discovered including SARS-CoV in 2003, HCoV NL63 in 2003, HCoV HKU1 in 2004, MERS-CoV in 2013, and SARS-CoV-2 in 2019. SARS-CoV-2 and most other coronaviruses cause serious respiratory infections on humans.

Coronavirus disease 2019 (COVID-19) is caused by severe acute respiratory syndrome coronavirus 2 (SARS‑CoV‑2) \cite{coronavirus_species}. It was reported in December 2019 in Wuhan, China as a pneumonia outbreak. In January 2020, the transmission of virus from human to human was confirmed, and the first quarantine was applied by Chinese Government in lately January 2020. On February 2020, the new coronavirus was named officially as SARS‑CoV‑2, and the World Health Organization (WHO) declared a pandemic on 11 March, 2020 \cite{who_pandemic_declaration}.

The most common symptoms of COVID-19 is known as fever, dry cough and tiredness. The loss of taste or smell, diarrhoea, headache, aches and pains, sore throat, and conjunctivitis are also agreed to strong high probability symptoms. Since the symptoms are very common with most other diseases, and they are not still precise, it is significant to find a highly accurate distinctive method to detect COVID-19. Moreover, infected people begin to show symptoms within an average of 5-6 days; however, this period can take up to 14 days \cite{who_qa}. Furthermore, the demographic information of a patient, i.e. the age and sex information, has a major role on how the patient is effected on COVID-19.

As of April 24, 2021, COVID-19 has caused more than 145 million people to be infected and more than 3 million deaths worldwide, and continues to spread rapidly \cite{covid19_news}. As stated above, the demographic information has a huge effect on death ratios. Anyways, early and rapid diagnosis plays a very important role in the treatment of the disease. That is why, WHO published protocols on diagnostic detection for COVID-19 on 13 and 17 January, 2020. Polymerase Chain Reaction (PCR) test \cite{pcr_cleveland_clinic} is the most basic method of COVID-19 diagnosis; however, the sensitivity is not high enough for the low viral load. Moreover, laboratory errors present in test samples. Because of it, all symptoms, test result and other reports must be considered on diagnostic process.

COVID-19, a respiratory disease, also plays a major role in the increase of pneumonia. Therefore, the diseases most confused with COVID-19 are those that cause pneumonia. One way to distinguish them is to examine the most damaged organ, lungs. Chest X-Rays are one of the fastest, the most accessible and the most reliable way to make observations on human lungs. However, the diagnosis of pneumonia from chest X-Rays may be a difficult task even for expert radiologists. The most important part of detecting COVID-19 by chest X-Rays is to distinguish the cause of pneumonia \cite{covid_vs_pneumonia}. Therefore, computerized support systems are needed to help radiologists diagnose pneumonia from chest X-Ray images.

Thanks to large data sets, rapid progress is being made in artificial intelligence-based computer vision problems. It has been observed that deep learning gives better results compared to manual diagnosis in problems such as object recognition, perception and segmentation. On the other hand, machine learning has various strong classification algorithms on categorical data including either small or large information. 

\section{Purpose of Thesis}\label{purposeofthesis}

The scope of this study is two fold: 1) to show the capability of deep features transfer learning from deep learning to machine learning, and 2) to apply this knowledge onto an ongoing global public health problem. Deep learning is performed for feature extraction with methodologies which are AlexNet, RestNet-18, Resnet-50, VGG16 and VGG19, and machine learning is performed for final classifications with algorithms including their original and regularized versions which are Support Vector Machines, Logistic Regression, K-Nearest Neighborhoods and Linear Discriminant Analysis. Moreover, the demographic information of patients in thesis dataset is also studied to make conclusions on their effect to machine learning classification algorithms.

For this purpose, chest X-Ray images shared on GitHub platform were used. In this dataset, there are data with and without demographic information, non-diseased, virus-induced pneumonia, bacterial-induced pneumonia, fungal-induced pneumonia, and lipoid-induced pneumonia. Moreover, samples with pneumonia finding have detailed causes such as COVID-19, Influenza, Escherichia coli, Aspergillosis spp., etc. Hence, data extraction on this dataset was applied to obtain two classes as non-COVID-19-diseased and COVID-19-diseased of data having demographic information.

By applying the stated scope to the mentioned dataset, the main aim of this thesis is to explain how to prepare data for study, how to perform deep learning algorithms, machine learning algorithms and transfer learning from deep learning to machine learning, how to use demographic information together with chest X-Ray image data, and to show how demographic information affects and what are the effects of regularization on machine learning algorithms by various performance measures.

\section{Literature Review}\label{literaturereview}

With the increase in information about COVID-19 and the formation of data that can be used during this period, artificial intelligence researchers had started to work on this disease. Since both the effectiveness and the limitations of machine learning and deep learning, which are the sub-branches of artificial intelligence, had been already experienced and known by the previous numerous studies, researchers could started new studies without wasting time and achieved satisfying results rapidly.

Ali Abbasian Ardakani et al. \cite{literature_ARDAKANI} worked on chest computed tomography (CT) images to detect whether a person has COVID-19 disease or not. They studied on 10 different convolutional neural network (CNN) models, and compared them. The CNN models they compared are AlexNet, VGG-16, VGG-19, SqueezeNet, GoogleNet, MobileNet-V2, ResNet-18, ResNet-50, ResNet-101, and Xception. They achieved to the best results with ResNet-101 CNN model as the AUC score of 0.994, the sensitivity of 100\%, the specificity of 99.02\%, the accuracy of 99.51\%, the precision of 99.03\%, and the negative predictive value of 100\%. Moreover, the results of other CNN architectures can be found on the original paper.

Y. Pathak et al. \cite{literature_Pathak} used a deep transfer learning technique on chest computed tomography (CT) images to classify non-COVID-19 and COVID-19 people. They used pre-trained ResNet-50 CNN architecture on ImageNet \cite{imagenet} dataset to train their classification problem, and 10-Fold cross-validation to prevent overfitting. Then, they obtained the testing accuracy as 93.02\%.

Tulin Ozturk et al. \cite{literature_OZTURK} developed a new CNN architecture, named as DarkCovidNet, to classify chest X-Ray images among COVID-19 and no-finding, and among COVID-19, no-findings and pneumonia not caused by COVID-19. They inspired by the DarkNet architecture \cite{yolo_darknet} and constructed theirs as consisting of seventeen convolution layers and one fully-connected layer. The final test result were achieved as the accuracy of 87.02\% for 3-class classification problem, while it was 98.08 for binary classification problem.

Yujin Oh et al. \cite{literature_oh} solved COVID-19, normal and pneumonia not caused by COVID-19 classification problem with first segmenting the chest X-Ray images and yielding extracted lung areas. Then, segmented images were classified patch-by-patch in ResNet-18 CNN architecture based model. For the final decision among patches, the majority voting method was used, and the results were obtained as the accuracy of 88.9\%, the sensitivity of 85.9\%, and the specificity of 96.4\%.

Elshennawy and Ibrahim \cite{literature_elshennawy} proposed four different deep learning models to solve 3-class classification problem. The dataset consist of chest X-Ray images and the three classes are no finding, bacterial pneumonia and COVID-19. A CNN model containing 4 convolutional layers and 3 fully-connected layers is proposed and trained from stretch, and the validation loss and accuracy were obtained as 0.3020 and 92.19\% respectively. A LSTM-CNN model containing one LSTM layer, 4 convolutional layers and 2 fully-connected layers were developed, and the final results were yielded as the loss of 0.5771 and the accuracy of 91.80\%. Furthermore, two CNN models, ResNet152V2 and MobileNetV2, were used as pre-trained. The test results for ResNet152V2 and MobileNetV2, were achieved as the loss of 0.0523 and the accuracy of 99.22\%, and as the loss of 0.1665 and the accuracy of 96.48\% respectively.

Ruaa Adeeb Al-Falluji et al. \cite{literature_Al-Falluji} were used a modified ResNet-18 CNN model to classify X-Ray images among COVID-19, no-findings and pneumonia not caused by COVID-19. The original ResNet-18 architecture was modified as changing the kernel size of conv1 convolution layer from 7 to 3, and adding two new convolution layers after the global average pooling layer. The final test result obtained with this technique was the accuracy of 96.73\%.

Majid Noue et al. \cite{A_novelCNNModel} was developed a novel CNN architecture, and solved the 3-class, COVID-19, normal and viral pneumonia other than SARS-CoV-2, based on Deep Features and Bayesian Optimization. The developed CNN architecture includes 5 convolution layers, 3 fully-connected layers and Softmax activation layer at the end. Deep features extracted from fc1 and fc2 layers are used to feed machine learning algorithm, which are support vector machine, decision tree and k-nearest neighbor, for final classification. The result achieved for features extracted from fc2 are obtained by SVM classifier as the sensitivity of 89.39\%, the specificity of 99.75\%, the F-score of 96.72, and the accuracy of 98.97\%.

The results of studies stated in this section can be viewed as merged in the Table~\ref{tab:literature_comparison}.

\begin{landscape}
\begin{table}[]
\centering
\caption{Reviewed works in the literature and their stated results.}
\label{tab:literature_comparison}
\begin{tabular}{lllcccc}
\hline
\multicolumn{1}{c}{\textbf{Authors}}  & \multicolumn{1}{c}{\textbf{Technique}}                                                                                       & \multicolumn{1}{c}{\textbf{Dataset}} & \textbf{\begin{tabular}[c]{@{}c@{}}The Number\\  of Classes\end{tabular}} & \textbf{\begin{tabular}[c]{@{}c@{}}Accuracy \\ (\%)\end{tabular}} & \textbf{\begin{tabular}[c]{@{}c@{}}Sensitivity\\ (\%)\end{tabular}} & \textbf{\begin{tabular}[c]{@{}c@{}}Specificity\\ (\%)\end{tabular}} \\ \hline \hline
Ali   Abbasian Ardakani et al. (2020) & ResNet-101 CNN model                                                                                                         & Chest CT images                      & 3                                                                         & 99.51                                                             & 100.00                                                              & 99.02                                                               \\ \hline
Y. Pathak et al. (2020)               & \begin{tabular}[c]{@{}l@{}}Deep transfer learning\\ on ResNet-50 CNN model\end{tabular}                                      & Chest CT images                      & 3                                                                         & 93.02                                                             & 91.46                                                               & 94.78                                                               \\ \hline
Tulin   Ozturk et al. (2020)          & DarkCovidNet CNN model                                                                                                       & Chest X-Ray images                   & 3                                                                         & 87.02                                                             & 92.18                                                               & 89.96                                                               \\ \hline
Tulin   Ozturk et al. (2020)          & DarkCovidNet CNN model                                                                                                       & Chest X-Ray images                   & 2                                                                         & 98.08                                                             & 95.13                                                               & 95.30                                                               \\ \hline
Yujin Oh et al. (2020)                & \begin{tabular}[c]{@{}l@{}}Patch-by-patch classification\\ in ResNet-18 CNN based\\ model after segmentation\end{tabular}    & Chest X-Ray images                   & 2                                                                         & 88.90                                                             & 85.90                                                               & 96.40                                                               \\ \hline
Elshennawy and Ibrahim (2020)         & \begin{tabular}[c]{@{}l@{}}Pre-trained ResNet152V2\\ CNN model\end{tabular}                                                  & Chest X-Ray images                   & 2                                                                         & 99.22                                                             & 99.44                                                               & 99.45                                                               \\ \hline
Ruaa Adeeb Al-Falluji et al. (2020)   & \begin{tabular}[c]{@{}l@{}}Deep transfer learning\\ on the modified ResNet-18 \\ CNN model\end{tabular}                      & Chest X-Ray images                   & 2                                                                         & 96.73                                                             & 94.00                                                               & 100.00                                                              \\ \hline
Majid   Noue et al. (2020)            & \begin{tabular}[c]{@{}l@{}}SVM classifier fed with deep\\ features obtained from the\\ novel proposed CNN model\end{tabular} & Chest X-Ray images                   & 2                                                                         & 98.97                                                             & 89.39                                                               & 99.75                                                               \\ \hline
\end{tabular}
\end{table}
\end{landscape}

\section{Structure}

This thesis consists of seven chapters. Chapter 1 presents an overview of the study including the history of COVID-19 disease, the scope and aim of thesis, the literature survey on related works, and this structure.

In Chapter 2, the information about chest X-Rays are given.

Chapter 3 introduces the basics of deep learning, loss functions and optimization methods, the basics of convolutional neural networks, transfer learning, and the CNN models experimented in this study such that AlexNet, ResNet-18, ResNet-34, ResNet-50, VGG16 and VGG19.

Chapter 4 introduces the basics of machine learning, the method of cross-validation, the regularization, and the ML algorithms experimented in this study such that SVM, LR, KNN and LDA.

In Chapter 5, all experiments on convolutional neural networks and machine learning carried out in this study take place together with how the dataset was constructed, data augmentation, deep feature extraction, how the feature matrices were formed, and the methods for hyper-parameter tuning on ML.

In Chapter 6, all results of experiments are given together with how we measure the performance by confusion matrices.

Finally, in Chapter 7, the conclusions of our study are presented with interpretations and suggestions for further researches.
